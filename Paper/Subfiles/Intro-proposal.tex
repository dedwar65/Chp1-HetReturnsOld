\input{./econtexRoot.texinput}
\documentclass[\econtexRoot/Chp1proposal]{subfiles}
\onlyinsubfile{\externaldocument{\econtexRoot/Chp1proposal}} % Get xrefs -- esp to apndx -- from main file; only works if main file has already been compiled

\begin{document}

\hypertarget{Introduction}{}
\section{Introduction}\notinsubfile{\label{sec:intro}}
\setcounter{page}{0}\pagenumbering{arabic}


\par The unequal distribution of wealth, an extensively documented phenomenon in numerous countries, has not only endured over time but also intensified in recent years. This point is stressed in a recent article from the Institute for Policy Studies (IPS), which revealed that in 2018, the total wealth of the poorest half of Americans was eclipsed by the combined wealth of the three wealthiest men in the nation. Additionally, the IPS report notes that the combined wealth of the top five richest men on this list skyrocketed by a staggering 123\% from March 2020 to October 2021\footnote{See Inequality.org articles data November 21, 2022: \say{Wealth Inequality in the United States} and \say{Updates: Billionaire Wealth, U.S. Job Losses and Pandemic Profiteers} (date accessed: March 27, 2023)}.

\par Wealth inequality has also been a subject of considerable interest throughout history in various academic fields. The statistics literature, for instance, focused on linking the distribution of income to the observable skewness in wealth distribution. The economics literature went further by establishing microfoundations for individual wealth outcomes.

\par More recently, the macroeconomics literature on inequality has seen significant growth, with the distribution of wealth among households offering insight into how the economy as a whole responds to aggregate fiscal shocks. This has been accomplished in large part due to the widespread adoption of models that abandon the traditional representative agent assumption. Specifically, a model that studies the equilibrium outcomes of an economy composed of individual decision-makers using a single aggregate agent can only have one marginal propensity to consume (MPC). As a result, in response to an aggregate fiscal shock, all households would respond similarly to a one-time stimulus check, which does not align with what transpired during the pandemic\footnote{\cite{jpjsledj2022} note that \say{In sum, while on average the [economic impact payments] EIPs appear to have gone to many households with incomes that were unharmed by the pandemic, some of the EIPs, mainly in the first round, did support short-term spending for some households, primarily those with low ex ante liquid wealth and those reliant on income that could not be earned by working from home.}}. 

\par The first departure from the representative agent framework entails positing an exogenously determined income process that generates a distribution of income among households. One can further assume that individuals face some level of potential unemployment in each period, creating a precautionary savings motive for consumers. The availability of a riskless asset that partially insures against this income risk results in households choosing to hold different levels of market resources optimally. \cite{ks1998}'s seminal work suggests that models assuming heterogeneity in individual income perform well in matching the aggregate capital stock but poorly in matching the distribution of wealth. 

\par The next step towards moving beyond the standard representative agent framework is to assume greater heterogeneity among households, leading more households to optimally hold lower levels of wealth. \cite{cstw2017} accomplish this by performing a structural estimation of \textit{ex-ante} heterogeneity amongst households which allows for the model's distribution of wealth to closely match the 2004 Survey of Consumer Finances (SCF) data on household wealth. Specifically, the authors assume different agents have different rates of time preference; this assumption of modest heterogeneity in time preferences is sufficient to match both the shape and skewness of the empirical distribution of wealth. Furthermore, while traditional representative agent models generate an aggregate marginal propensity to consume between $0.02$ and $0.04$, the $\beta$-dist model generates an aggregate MPC between $0.2$ and $0.4$. This range falls within the values estimated across households in the data.

\par It is worth noting that the time preference factor $(\beta)$ is one of the key parameters that influences an individual's equilibrium target level of market resources, but it is not directly observable. On the other hand, estimating differences in the rate of return to financial assets across households is possible, as this variable \textit{is} directly observable. Empirical research has been conducted to estimate such differences, with a recent example being the work of \cite{aflgdmlp20}. They analyzed 12 years of administrative tax records on capital income and wealth stock for all taxpayers in Norway from 2004-2015 to estimate these rates of return.

\par This paper aims to contribute the computational HA framework by performing a similar structural estimation exercise regarding ex-ante heterogeneity in the rate of return and comparing it to the recent, motivating empirical evidence on disparities in rates of return among households. Since the rate of return has a direct role in the wealth accumulation process, differences in returns to assets across households may be a compelling explanation for wealth inequality.



\onlyinsubfile{\input{\LaTeXInputs/bibliography_blend}}

\ifthenelse{\boolean{Web}}{}{
  \onlyinsubfile{\captionsetup[figure]{list=no}}
  \onlyinsubfile{\captionsetup[table]{list=no}}
}

\end{document} \endinput
