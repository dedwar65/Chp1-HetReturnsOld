\input{./econtexRoot.texinput}
\documentclass[\econtexRoot/Chp1proposal]{subfiles}
\onlyinsubfile{\externaldocument{\econtexRoot/Chp1proposal}} % Get xrefs -- esp to apndx -- from main file; only works if main file has already been compiled

\begin{document}

\hypertarget{Introduction}{}
\section{Introduction}\notinsubfile{\label{sec:intro}}
\setcounter{page}{0}\pagenumbering{arabic}

% \par The unequal distribution of wealth, an extensively documented historical phenomenon, has intensified in recent years. This point is stressed in a recent article from the Institute for Policy Studies (IPS) which revealed that, in 2018, the total wealth of the poorest half of Americans was eclipsed by the combined wealth of the three wealthiest men in the nation. The IPS report further states that the combined wealth of the top five richest men on this list skyrocketed by a staggering 123\% from March 2020 to October 2021\footnote{See Inequality.org articles data November 21, 2022: \say{Wealth Inequality in the United States} and \say{Updates: Billionaire Wealth, U.S. Job Losses and Pandemic Profiteers} (date accessed: March 27, 2023)}.

% \par Wealth inequality has also been a subject of considerable interest throughout history in various academic fields. The early statistics literature focused on linking the distribution of income to the observable skewness in wealth distribution. 

\par The macroeconomics literature has expanded to allow for quantitative statements to be made about the relationship between inequality and the economy in recent recent years. One example of this is the finding that the distribution of wealth across households offers insight into how the economy as a whole responds to aggregate fiscal shocks.\footnote{\cite{jpjsledj2022} note that \say{In sum, while on average the [economic impact payments] EIPs appear to have gone to many households with incomes that were unharmed by the pandemic, some of the EIPs, mainly in the first round, did support short-term spending for some households, primarily those with low ex ante liquid wealth and those reliant on income that could not be earned by working from home.}} Analysis of this sort has been accomplished in large part due to the widespread adoption of models that abandon the traditional representative agent assumption. 

\par The first departure from the representative agent framework incorporates labor income risk. There is a precautionary savings motive in this setting. The availability of a riskless asset that partially insures against this income risk results in households choosing to hold different levels of market resources optimally. \cite{ks1998}'s seminal work suggests that models assuming heterogeneity in individual income perform well in matching the aggregate capital stock but poorly in matching the distribution of wealth. 

\par The next departure is to assuming heterogeneity among households beyond the ex-post realizations of the stochastic process for income. This will lead to more households to optimally hold lower levels of wealth. \cite{cstw2017} accomplish this by performing a structural estimation of \textit{ex-ante} heterogeneity in time preferences which allows for the model's distribution of wealth to match the household wealth data much better. 

% Specifically, the authors assume different agents have different rates of time preference; this assumption of modest heterogeneity in time preferences is sufficient to match both the shape and skewness of the empirical distribution of wealth. Furthermore, while traditional representative agent models generate an aggregate marginal propensity to consume between $0.02$ and $0.04$, the $\beta$-dist model generates an aggregate MPC between $0.2$ and $0.4$. This range falls within the values estimated across households in the data.

% \subsection{Contributions to the literature}

\par Although the time preference factor $(\beta)$ is a key parameter in determining a household's target level of market resources, it is not directly observable. The rate of return to, on the other hand, \textit{is} directly observable and can be estimated. Recently, \cite{aflgdmlp20} analyze 12 years of administrative tax records on capital income and wealth stock for all taxpayers in Norway from 2004-2015 to estimate these rates of return. This serves as motivation for the HA model of heterogeneous returns which I present in this paper.

% \par Motivated by this recent evidence, this proposal provides (i) preliminary results and (ii) actionable extensions of a structural estimation exercise regarding ex-ante heterogeneity in the rate of return. Since the rate of return has a direct role in the wealth accumulation process, differences in returns to assets across households may be a compelling explanation for wealth inequality.

\onlyinsubfile{\input{\LaTeXInputs/bibliography_blend}}

\ifthenelse{\boolean{Web}}{}{
  \onlyinsubfile{\captionsetup[figure]{list=no}}
  \onlyinsubfile{\captionsetup[table]{list=no}}
}

\end{document} \endinput
