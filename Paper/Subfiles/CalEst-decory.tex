\input{./econtexRoot.texinput}
\documentclass[\econtexRoot/Preproposal]{subfiles}
\onlyinsubfile{\externaldocument{\econtexRoot/Preproposal}} % Get xrefs -- esp to apndx -- from main file; only works if main file has already been compiled

\begin{document}

\hypertarget{CalEst}{}
\onlyinsubfile{\setcounter{section}{3}}
\section{Calibration and Estimating Heterogeneity in the Rate of Return}\notinsubfile{\label{sec:sim}}


\par As this note is a proposal for work to come, this section will be devoted to the practical work involved with completing the project. The first step is to properly calibrate the model using parameter values from the perpetual youth model literature. From there, the next step in extending the model is to make an assumption regarding the ex-ante distribution of household rates of return. In this way,I am after the '$R$-Dist' version of the model (again, the distribution may not be uniform on some interval, but it is an initial step). After this, I will simulate the model and use wealth shares data to find the \textit{Lorenz distance}; that is, the values of $\grave{R}$ and $\nabla$ that solve the following:

$$ \{\grave{R}, \nabla\} = \text{arg}\min_{R, \nabla} \bigg( \sum_{i=20, 40, 60, 80} (w_{i}(R, \nabla)-\omega_i )^{2} \bigg)^{\frac{1}{2}}. $$

\par The last step in the main portion of this project, assuming that this extension is able to match the observed distribution of wealth reasonably well, is to check whether or not the model gives reasonable predictions regarding the aggregate MPC in response to a one-time fiscal shock. As in the \cite{cstw2017} work, we are interested in the scenario where the economy is in its steady-state equilibrium up to some date $t$. Given an announcement of a one-time stimulus check from the government \textit{before} consumption decisions are made in period $t$, the final task is to then answer: how much much will aggregate consumption increase?


\onlyinsubfile{% Allows two (optional) supplements to hard-wired \texname.bib bibfile:
% system.bib is a default bibfile that supplies anything missing elsewhere
% Add-Refs.bib is an override bibfile that supplants anything in \texfile.bib or system.bib
\provideboolean{AddRefsExists}
\provideboolean{systemExists}
\provideboolean{BothExist}
\provideboolean{NeitherExists}
\setboolean{BothExist}{true}
\setboolean{NeitherExists}{true}

\IfFileExists{\econtexRoot/Add-Refs.bib}{
  % then
  \typeout{References in Add-Refs.bib will take precedence over those elsewhere}
  \setboolean{AddRefsExists}{true}
  \setboolean{NeitherExists}{false} % Default is true
}{
  % else
  \setboolean{AddRefsExists}{false} % No added refs exist so defaults will be used
  \setboolean{BothExist}{false}     % Default is that Add-Refs and system.bib both exist
}

% Deal with case where system.bib is found by kpsewhich
\IfFileExists{/usr/local/texlive/texmf-local/bibtex/bib/system.bib}{
  % then
  \typeout{References in system.bib will be used for items not found elsewhere}
  \setboolean{systemExists}{true}
  \setboolean{NeitherExists}{false}
}{
  % else
  \typeout{Found no system database file}
  \setboolean{systemExists}{false}
  \setboolean{BothExist}{false}
}

\ifthenelse{\boolean{showPageHead}}{ %then
  \clearpairofpagestyles % No header for references pages
  }{} % No head has been set to clear

\ifthenelse{\boolean{BothExist}}{
  % then use both
  \typeout{bibliography{\econtexRoot/Add-Refs,\econtexRoot/\texname,system}}
  \bibliography{\econtexRoot/Add-Refs,\econtexRoot/\texname,system}
  % else both do not exist
}{ % maybe neither does?
  \ifthenelse{\boolean{NeitherExists}}{
    \typeout{bibliography{\texname}}
    \bibliography{\texname}}{
    % no -- at least one exists
    \ifthenelse{\boolean{AddRefsExists}}{
      \typeout{bibliography{\econtexRoot/Add-Refs,\econtexRoot/\texname}}
      \bibliography{\econtexRoot/Add-Refs,\econtexRoot/\texname}}{
      \typeout{bibliography{\econtexRoot/\texname,system}}
      \bibliography{        \econtexRoot/\texname,system}}
  } % end of picking the one that exists
} % end of testing whether neither exists
}

\ifthenelse{\boolean{Web}}{}{
  \onlyinsubfile{\captionsetup[figure]{list=no}}
  \onlyinsubfile{\captionsetup[table]{list=no}}
}

\end{document} \endinput
