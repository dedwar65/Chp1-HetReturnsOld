\input{./econtexRoot.texinput}
\documentclass[\econtexRoot/Chp1proposal]{subfiles}
\onlyinsubfile{\externaldocument{\econtexRoot/Chp1proposal}} % Get xrefs -- esp to apndx -- from main file; only works if main file has already been compiled

\begin{document}

\onlyinsubfile{\setcounter{section}{2}}
\section{Incorporating life cycle dynamics into the model}
\notinsubfile{\label{sec:Model}}

\par More realistic assumptions regarding the age and education level of households can have important implications for the income and mortality process of households. Here, I extend the model to incorporate these life cycle dynamics.

\par Households enter the economy at time $t$ aged 24 years old and are endowed with an education level $e \in \{D,HS,C\}$, and initial permanent income level $\textbf{p}_0$, and a capital stock $k_0$. The life cycle version of household income is given by:

$$ y_t = \xi_t \textbf{p}_t = (1 - \tau) \theta_t \textbf{p}_t, $$

where $\textbf{p}_t = \psi_t \bar{\psi}_{es} \textbf{p}_{t-1}$ and $\bar{\psi}_{es}$ captures the age-education-specific average growth factor. Households that have lived for $s$ periods have permanent shocks drawn from a lognormal distribution with mean $1$ and variance $\sigma^{2}_{\psi s}$ and transitory shocks drawn from a lognormal distribution with mean $\frac{1}{\cancel{\mho}}$ and variance $\sigma^{2}_{\theta s}$ with probability $\cancel{\mho} = (1-\mho)$ and $\mu$ with probability $\mho$.

\par The normalized version of the age-education-specific consumption-saving problem for households is given by

\begin{eqnarray*}
  v_{es}(m_t) &=& \max_{c_t} u(c_t(m_t)) + \beta \cancel{D}_{es} \mathbb{E}_{t}[\psi_{t+1}^{1-\rho}v_{es + 1}(m_{t+1})] \\
  &\text{s.t.}& \\
  a_t &=& m_t - c_t, \\
  k_{t+1} &=& \frac{a_t}{\psi_{t+1}}, \\
  m_{t+1} &=& (\daleth + r_t)k_{t+1} + \xi_{t+1}, \\
  a_t &\geq& 0.
\end{eqnarray*}

\subsection{Results}

\par The additional parameters necessary to calibrate the life cycle version of the model are given in table \ref{tab:calib2}.

\hypertarget{calibLC}{}
\begin{table}[ht]
  \centering
  \resizebox{0.6\textwidth}{!}{
    \begin{tabular}{ccc}
        \toprule
        Description & Parameter & Value  \\
        \midrule
        Population growth rate & $N$ & 0.0025  \\
        Technological growth rate & $\Gamma$ & 0.0037  \\
        Rate of high school dropouts & $\theta_D $ & 0.11  \\
        Rate of high school graduates & $\theta_{HS} $ & 0.55  \\
        Rate of college graduates & $\theta_C $ & 0.34  \\
        Labor income tax rate & $\tau$ & 0.0942  \\
        \bottomrule
    \end{tabular}}
    \caption{Parameter values (annual frequency) for the lifecycle model.}
    \label{tab:calib2}
\end{table}

\unskip

\par The estimation procedure finds this optimal value to be $\Rfree = 1.0626$ for the R-point model in this setting. The estimation procedure for the R-dist model in the life cycle setting finds optimal values of $\Rfree = 1.0395$ and $\nabla = 0.0737$. Notice the improved performance of the estimation in matching the data displayed in figure \ref{fig:LCUnif}.

 \hypertarget{LCUnif}{}
 \begin{figure}[H]
   \centering
   \includegraphics[width=0.8\textwidth]{./Figures/LCUnif.png}
   \caption{Life cycle lorenz curve v.s. data}
    \label{fig:LCUnif}
  \end{figure}\unskip

\onlyinsubfile{\input{\LaTeXInputs/bibliography_blend}}

\ifthenelse{\boolean{Web}}{}{
  \onlyinsubfile{\captionsetup[figure]{list=no}}
  \onlyinsubfile{\captionsetup[table]{list=no}}
}
\end{document}	\endinput

