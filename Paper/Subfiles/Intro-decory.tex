\input{./econtexRoot.texinput}
\documentclass[\econtexRoot/Chp1]{subfiles}
\onlyinsubfile{\externaldocument{\econtexRoot/Chp1}} % Get xrefs -- esp to apndx -- from main file; only works if main file has already been compiled

\begin{document}

\hypertarget{Introduction}{}
\section{Introduction}\notinsubfile{\label{sec:intro}}
\setcounter{page}{0}\pagenumbering{arabic}


\par The unequal distribution of wealth is an extensively documented phenomenon in numerous countries. Regrettably, this feature has not only endured over time but also intensified in recent years. This point is stressed in a recent article from the Institute for Policy Studies (IPS), which revealed that in 2018, the total wealth of the poorest half of Americans was eclipsed by the combined wealth of the three wealthiest men in the nation. The term \say{richest} denotes one's standing in Forbes magazine's list of the 400 richest individuals. Additionally, the IPS report notes that the combined wealth of the top five richest men on this list skyrocketed by a staggering 123\% from March 2020 to October 2021\footnote{See Inequality.org articles data November 21, 2022: \say{Wealth Inequality in the United States} and \say{Updates: Billionaire Wealth, U.S. Job Losses and Pandemic Profiteers} (date accessed: March 27, 2023)}. 

\par The unequal distribution of wealth has also been a subject of considerable interest throughout history in various fields. The statistics literature, for instance, focused on linking the distribution of income to the observable skewness in wealth distribution. The economics literature went further by establishing microfoundations for individual wealth outcomes. Similarly, the macroeconomics literature on inequality has seen significant growth, with the distribution of wealth among households offering insight into how the economy as a whole responds to aggregate fiscal shocks. The recent stimulus checks issued during the pandemic serve as a timely example of this phenomenon.

\par The macroeconomics literature has undergone significant changes in recent years, with the widespread adoption of models that abandon the traditional representative agent assumption in their analysis. Specifically, a model that studies the equilibrium outcomes of an economy composed of individual decision-makers using a single aggregate agent can only have one marginal propensity to consume (MPC). As a result, in response to an aggregate fiscal shock, all households would respond similarly to a one-time stimulus check, which does not align with what transpired during the pandemic\footnote{\cite{jpjsledj2022} note that \say{In sum, while on average the [economic impact payments] EIPs appear to have gone to many households with incomes that were unharmed by the pandemic, some of the EIPs, mainly in the first round, did support short-term spending for some households, primarily those with low ex ante liquid wealth and those reliant on income that could not be earned by working from home.}}. Heterogeneous agent models have emerged as a prominent alternative, offering a more accurate representation of the diversity of economic behaviors and outcomes among households.

\par The first departure from the representative agent framework entails positing an exogenously determined income process that generates a distribution of income among households. One common approach to incorporating heterogeneity is to adopt \cite{mf1957}'s description of a permanent and transitory component in the income process. To account for business cycle dynamics, one can further assume that individuals face some level of potential unemployment in each period, creating a precautionary savings motive for consumers. Given that such uncertainty cannot be fully insured against, the availability of a riskless asset that partially insures against income risk results in households choosing to hold different levels of market resources optimally.

\par  \cite{ks1998}'s seminal work suggests that models assuming heterogeneity in individual income perform well in matching the aggregate capital stock but poorly in matching the distribution of wealth. The resulting optimal consumption function is concave in an individual's wealth holdings, meaning that the marginal propensity to consume out of income is increasingly higher at lower levels of wealth. Therefore, a model that places too many households in the middle of the wealth distribution relative to those at lower levels will struggle to match the average MPC estimated from household data. Since our focus is on the implications of fiscal policy for the entire economy, a macroeconomic model's failure to match the observed wealth distribution in its implied equilibrium is significant. 

\par Moving beyond the standard representative agent framework, the next step is to assume greater heterogeneity among households, leading more households to optimally hold lower levels of wealth. \cite{gkgv22}'s  recent work provides a comprehensive survey of models that reject this assumption, instead utilizing heterogeneous agent, incomplete markets models featuring (i) uninsurable idiosyncratic income risk, (ii) a precautionary savings motive, and (iii) an endogenous wealth distribution.

\par  \cite{cstw2017} adopt this approach and further extend the baseline setting to allow for ex-ante heterogeneity amongst households. Specifically, they assume different agents have different rates of time preference, which reflects implicit characteristics of households relevant to their lifetime wealth accumulation. The authors find that this assumption of modest heterogeneity in time preferences is sufficient to match both the shape and skewness of the empirical distribution of wealth. Furthermore, while traditional representative agent models generate an aggregate marginal propensity to consume between $0.02$ and $0.04$, the $\beta$-dist model generates an aggregate MPC between $0.2$ and $0.4$. This range falls within the values estimated across households in the data.

\par The household's optimal consumption-savings problem contains additional elements that could contribute to disparities in wealth accumulation over the course of one's lifetime. It is worth noting that the time preference factor $(\beta)$ is one of the key parameters that influences an individual's equilibrium target level of market resources, but it is not directly observable. Therefore, in order to estimate $\beta$, one would need to gather data through surveys or other methods that allow for the direct acquisition of information from households.

\par On the other hand, estimating differences in the rate of return to financial assets across households is possible, as this variable \textit{is} directly observable. Empirical research has been conducted to estimate such differences, with a recent example being the work of \cite{aflgdmlp20}. They analyzed 12 years of administrative tax records on capital income and wealth stock for all taxpayers in Norway from 2004-2015 to estimate these rates of return.


\par This paper aims to enhance the computational, heterogeneous agent modelling framework by integrating recent empirical evidence on disparities in rates of return among households. The objective is to better align the observed wealth distribution with the model predictions, thereby generating more realistic estimates of the average marginal propensity to consume among households.



\onlyinsubfile{% Allows two (optional) supplements to hard-wired \texname.bib bibfile:
% system.bib is a default bibfile that supplies anything missing elsewhere
% Add-Refs.bib is an override bibfile that supplants anything in \texfile.bib or system.bib
\provideboolean{AddRefsExists}
\provideboolean{systemExists}
\provideboolean{BothExist}
\provideboolean{NeitherExists}
\setboolean{BothExist}{true}
\setboolean{NeitherExists}{true}

\IfFileExists{\econtexRoot/Add-Refs.bib}{
  % then
  \typeout{References in Add-Refs.bib will take precedence over those elsewhere}
  \setboolean{AddRefsExists}{true}
  \setboolean{NeitherExists}{false} % Default is true
}{
  % else
  \setboolean{AddRefsExists}{false} % No added refs exist so defaults will be used
  \setboolean{BothExist}{false}     % Default is that Add-Refs and system.bib both exist
}

% Deal with case where system.bib is found by kpsewhich
\IfFileExists{/usr/local/texlive/texmf-local/bibtex/bib/system.bib}{
  % then
  \typeout{References in system.bib will be used for items not found elsewhere}
  \setboolean{systemExists}{true}
  \setboolean{NeitherExists}{false}
}{
  % else
  \typeout{Found no system database file}
  \setboolean{systemExists}{false}
  \setboolean{BothExist}{false}
}

\ifthenelse{\boolean{showPageHead}}{ %then
  \clearpairofpagestyles % No header for references pages
  }{} % No head has been set to clear

\ifthenelse{\boolean{BothExist}}{
  % then use both
  \typeout{bibliography{\econtexRoot/Add-Refs,\econtexRoot/\texname,system}}
  \bibliography{\econtexRoot/Add-Refs,\econtexRoot/\texname,system}
  % else both do not exist
}{ % maybe neither does?
  \ifthenelse{\boolean{NeitherExists}}{
    \typeout{bibliography{\texname}}
    \bibliography{\texname}}{
    % no -- at least one exists
    \ifthenelse{\boolean{AddRefsExists}}{
      \typeout{bibliography{\econtexRoot/Add-Refs,\econtexRoot/\texname}}
      \bibliography{\econtexRoot/Add-Refs,\econtexRoot/\texname}}{
      \typeout{bibliography{\econtexRoot/\texname,system}}
      \bibliography{        \econtexRoot/\texname,system}}
  } % end of picking the one that exists
} % end of testing whether neither exists
}

\ifthenelse{\boolean{Web}}{}{
  \onlyinsubfile{\captionsetup[figure]{list=no}}
  \onlyinsubfile{\captionsetup[table]{list=no}}
}

\end{document} \endinput
