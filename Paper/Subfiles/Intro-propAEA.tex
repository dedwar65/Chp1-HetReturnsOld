\input{./econtexRoot.texinput}
\documentclass[\econtexRoot/Chp1proposal]{subfiles}
\onlyinsubfile{\externaldocument{\econtexRoot/Chp1proposal}} % Get xrefs -- esp to apndx -- from main file; only works if main file has already been compiled

\begin{document}

\hypertarget{Introduction}{}
\section{Introduction}\notinsubfile{\label{sec:intro}}
\setcounter{page}{0}\pagenumbering{arabic}

\par The recent heterogeneous agent (HA) macro literature has managed to construct models that are both microeconomically realistic in terms of household financial choices. Such models, either in an infinite horizon context or using a life cycle specification, are able to match measures of wealth inequality by assuming heterogeneity in time preference rates across agents (\cite{cstw2017}).

\par This literature makes the traditional assumption that all households earn the same rate of return for publicly available assets (like bank accounts or stock investments). But newly available estimates from Norwegian registry data find that, in fact, there are large differences across households in rates of return even within narrowly defined categories of assets (\cite{aflgdmlp20}). %This notable finding has implications for the management of consumer finances: it is well known that differences in returns to assets will lead to a skewed distribution of wealth\footnote{\cite{jbab18} provide a useful survey of the wealth inequality literature.}.

\par My research agenda is to understand both the consequences of these differences in rates of return, and their causes. Work that I've accomplished so far has found that when heterogeneity in rates of return consistent with the Norwegian data are substituted for the usual assumption of homogeneous rates of return, time preference heterogeneity is no longer necessary for such models to match the observed degree of inequality (in either the infinite horizon or the life cycle specification of the model).

\section{Results}

\par Specifically, I use simulated method of moments (SMM) estimation to find the rate of return which minimzes the distances between wealth shares measured in Survey of Consumer Finances (SCF) and wealth shares which arise from solving a standard, infinite-horizon consumption-saving model. In the case where each household earns the same return to their savings, the estimate value of the rate of return is 1.071 using the 2004 SCF data. 

\par From there, I allow for differences in the rate of rate earned on assets by assuming that there is a uniform distribution of returns across households. In this case, the mean and standard deviation of that estimated distribution are 1.055 and .006, respectively.

\par I reestimate the model for the life-cycle version of the model and find that the estimated value of the rate of return is 1.063 when households are restricted to earning the same return on their assets. When I allow for heterogeneous returns in the life-cycle setting, the estimated uniform distribution has a mean of 1.039 and a standard deviation of 1.012.

\section{Next Steps}

\par The next step for this project is to see how much my estimates for the optimal rate of return (with and without heterogeneity) changes for other plausible values for other parameters of interest to the consumption-saving problem. For example, empirical estimates of the time preference factor are known to vary substantially across methodologies used, but can typically be found in the range of $1$\% to $10$\%. Estimates of the CRRA factor in similar frameworks are generally between $1$ and $4$.



\onlyinsubfile{% Allows two (optional) supplements to hard-wired \texname.bib bibfile:
% system.bib is a default bibfile that supplies anything missing elsewhere
% Add-Refs.bib is an override bibfile that supplants anything in \texfile.bib or system.bib
\provideboolean{AddRefsExists}
\provideboolean{systemExists}
\provideboolean{BothExist}
\provideboolean{NeitherExists}
\setboolean{BothExist}{true}
\setboolean{NeitherExists}{true}

\IfFileExists{\econtexRoot/Add-Refs.bib}{
  % then
  \typeout{References in Add-Refs.bib will take precedence over those elsewhere}
  \setboolean{AddRefsExists}{true}
  \setboolean{NeitherExists}{false} % Default is true
}{
  % else
  \setboolean{AddRefsExists}{false} % No added refs exist so defaults will be used
  \setboolean{BothExist}{false}     % Default is that Add-Refs and system.bib both exist
}

% Deal with case where system.bib is found by kpsewhich
\IfFileExists{/usr/local/texlive/texmf-local/bibtex/bib/system.bib}{
  % then
  \typeout{References in system.bib will be used for items not found elsewhere}
  \setboolean{systemExists}{true}
  \setboolean{NeitherExists}{false}
}{
  % else
  \typeout{Found no system database file}
  \setboolean{systemExists}{false}
  \setboolean{BothExist}{false}
}

\ifthenelse{\boolean{showPageHead}}{ %then
  \clearpairofpagestyles % No header for references pages
  }{} % No head has been set to clear

\ifthenelse{\boolean{BothExist}}{
  % then use both
  \typeout{bibliography{\econtexRoot/Add-Refs,\econtexRoot/\texname,system}}
  \bibliography{\econtexRoot/Add-Refs,\econtexRoot/\texname,system}
  % else both do not exist
}{ % maybe neither does?
  \ifthenelse{\boolean{NeitherExists}}{
    \typeout{bibliography{\texname}}
    \bibliography{\texname}}{
    % no -- at least one exists
    \ifthenelse{\boolean{AddRefsExists}}{
      \typeout{bibliography{\econtexRoot/Add-Refs,\econtexRoot/\texname}}
      \bibliography{\econtexRoot/Add-Refs,\econtexRoot/\texname}}{
      \typeout{bibliography{\econtexRoot/\texname,system}}
      \bibliography{        \econtexRoot/\texname,system}}
  } % end of picking the one that exists
} % end of testing whether neither exists
}

\ifthenelse{\boolean{Web}}{}{
  \onlyinsubfile{\captionsetup[figure]{list=no}}
  \onlyinsubfile{\captionsetup[table]{list=no}}
}

\end{document} \endinput
