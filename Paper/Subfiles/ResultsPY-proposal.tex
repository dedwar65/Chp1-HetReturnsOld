\input{./econtexRoot.texinput}
\documentclass[\econtexRoot/Chp1proposal]{subfiles}
\onlyinsubfile{\externaldocument{\econtexRoot/Chp1proposal}} % Get xrefs -- esp to apndx -- from main file; only works if main file has already been compiled

\begin{document}

\hypertarget{Results}{}
\section{Results}\notinsubfile{\label{sec:results}}
%\setcounter{page}{0}\pagenumbering{arabic}

\par To solve and simulate the model, I follow the calibration scheme captured in the table below.
\input{./Tables/calibPY.tex}\unskip

\subsection{The model without heterogeneity}

\par The solution of the model with no heterogeneity in returns (referred to as the R-point model) is the one which finds the value for the rate of return $\Rfree$ which minimizes the distance between the simulated and empirical wealth shares at the 20th, 40th, 60th, and 80th percentiles of the corresponding wealth distribution. The estimation procedure finds this optimal value to be $\Rfree = 1.0153$.

\subsection{Incorporating heterogeneous returns}

\par Recent studies by \cite{aflgdmlp20} and \cite{lblcps18} have not only estimated the rate of return on asset holdings but have also uncovered significant heterogeneity across households. Given this motivation, the revised model assumes the existence of multiple types of agents, each earning a distinct rate of return on their assets.

\par Specifically, I assume that different types of households have a time preference factor drawn uniformly from the interval $(\grave{\Rfree} - \nabla, \grave{\Rfree} + \nabla)$, where $\nabla$ represents the level of dispersion. Afterward, the model is simulated to estimate the values of both $\grave{\Rfree}$ and $\nabla$ so that the model matches the inequality in the wealth distribution. To achieve this, the following minimization problem is solved:

$$ \{\grave{\Rfree}, \nabla\} = \text{arg}\min_{\Rfree, \nabla} \bigg( \sum_{i=20, 40, 60, 80} (w_{i}(\Rfree, \nabla)-\omega_i )^{2} \bigg)^{\frac{1}{2}} $$

\par subject to the constraint that the aggregate capital-to-output ratio in this model matches that of the perfect foresight setting:

$$ \frac{K}{Y} = \frac{K_{PF}}{Y_{PF}}. $$

\par Note that $w_i$ and $\omega_i$ give the porportion of total aggregate net worth held by the top $i$ percent in the model and in the data, respectively.

\par The estimation procedure finds this optimal values of $\Rfree = 1.0106$ and $\nabla = 0.0112$. The performance of the estimation of both the R-point and R-dist models, measured by their ability to match the 2004 SCF wealth data, is compared in figure \ref{fig:PYUnif}.

 \hypertarget{PYUnif}{}
 \begin{figure}[H]
   \centering
    \includegraphics[width=0.7\textwidth]{./Figures/PYUnif.png}
    %
    \caption{Perpetual youth lorenz curve v.s. data}
    \label{fig:PYUnif}
  \end{figure}


\onlyinsubfile{\input{\LaTeXInputs/bibliography_blend}}

\ifthenelse{\boolean{Web}}{}{
  \onlyinsubfile{\captionsetup[figure]{list=no}}
  \onlyinsubfile{\captionsetup[table]{list=no}}
}

\end{document} \endinput
