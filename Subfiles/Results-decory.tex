\input{./econtexRoot.texinput}
\documentclass[\econtexRoot/PreproposalFinal]{subfiles}
\onlyinsubfile{\externaldocument{\econtexRoot/PreproposalFinal}} % Get xrefs -- esp to apndx -- from main file; only works if main file has already been compiled

\begin{document}

\hypertarget{Results}{}
\section{Results}\notinsubfile{\label{sec:results}}
%\setcounter{page}{0}\pagenumbering{arabic}

\subsection{Matching observed inequality in the distribtution of wealth}

\subsubsection{Adding ex-ante heterogeneity in time preferences}

\par In \cite{cstw2017}'s baseline model, heterogeneity in the time preference factor is not accounted for. However, to address this, the model is extended to include this factor. This is done by assuming that different types of households have a time preference factor drawn uniformly from the interval $(\grave{\beta} - \nabla, \grave{\beta} + \nabla)$, where $\nabla$ represents the level of dispersion. Afterward, the model is simulated to estimate the values of both $\grave{\beta}$ and $\nabla$ so that the model matches the inequality in the wealth distribution. To achieve this, the following minimization problem is solved:

$$ \{\grave{\beta}, \nabla\} = \text{arg}\min_{\beta, \nabla} \bigg( \sum_{i=20, 40, 60, 80} (w_{i}(\beta, \nabla)-\omega_i )^{2} \bigg)^{\frac{1}{2}} $$

\par subject to the constraint that the aggregate capital-to-output ratio in this model matches that of the perfect foresight setting:

$$ \frac{K}{Y} = \frac{K_{PF}}{Y_{PF}}. $$

\par Note that $w_i$ and $\omega_i$ give the porportion of total aggregate net worth held by the top $i$ percent in the model and in the data, respectively.

\subsubsection{The analogous exercise for ex-ante heterogenous rates of return}

\par The $\beta$-dist model proves to be useful in a setting where there are heterogeneous time preference factors since it captures an unobservable component of a household's decision-making process. While the microeconomics literature has put in considerable effort to estimate this parameter, there is currently no consensus on its value.

\par Recent studies by \cite{aflgdmlp20} and \cite{lblcps18} have not only estimated the rate of return on asset holdings but have also uncovered significant heterogeneity across households. Given this motivation, the revised model assumes the existence of multiple types of agents, each earning a distinct rate of return on their assets. A calibration exercise akin to the one used in the $\beta$-dist model is then performed. This crucial step involves comparing the resulting endogenous distribution from simulating this calibrated model to its empirical counterpart to determine if there is an ex-ante distribution of rates of return that can match the observable inequality in the wealth distribution. If a distribution of returns to asset holdings satisfies this criterion, the final step involves reconciling this model heterogeneity with the observed differences in rates of return found in the aforementioned literature.

\subsection{Policy implications}

\par Following the analysis of the wealth distribution, the policy implications of heterogeneous rates of return will be examined by evaluating the impact of a one-time stimulus payment to all households on key macroeconomic variables (such as consumption, capital, and output).

\par My suspicion is that if the model provides a better match to the inequality in the distribution of wealth compared to the case where all households earn the same rate of return on their assets, there will be greater dispersion in the marginal propensity to consume across households. Consequently, this should result in different consumption and saving behavior among households in response to a one-time shock to income. As a result, the aggregate implications of the policy shock in this setting are expected to differ significantly from the setting where there is no heterogeneity in the rate of return.

\par I aim to compare the policy implications of this computational model with the actual consumption and saving behavior of households during the recent pandemic, which was characterized by the provision of several stimulus checks by the government. However, I am currently concerned about the feasibility of this final step in the project.


\onlyinsubfile{% Allows two (optional) supplements to hard-wired \texname.bib bibfile:
% system.bib is a default bibfile that supplies anything missing elsewhere
% Add-Refs.bib is an override bibfile that supplants anything in \texfile.bib or system.bib
\provideboolean{AddRefsExists}
\provideboolean{systemExists}
\provideboolean{BothExist}
\provideboolean{NeitherExists}
\setboolean{BothExist}{true}
\setboolean{NeitherExists}{true}

\IfFileExists{\econtexRoot/Add-Refs.bib}{
  % then
  \typeout{References in Add-Refs.bib will take precedence over those elsewhere}
  \setboolean{AddRefsExists}{true}
  \setboolean{NeitherExists}{false} % Default is true
}{
  % else
  \setboolean{AddRefsExists}{false} % No added refs exist so defaults will be used
  \setboolean{BothExist}{false}     % Default is that Add-Refs and system.bib both exist
}

% Deal with case where system.bib is found by kpsewhich
\IfFileExists{/usr/local/texlive/texmf-local/bibtex/bib/system.bib}{
  % then
  \typeout{References in system.bib will be used for items not found elsewhere}
  \setboolean{systemExists}{true}
  \setboolean{NeitherExists}{false}
}{
  % else
  \typeout{Found no system database file}
  \setboolean{systemExists}{false}
  \setboolean{BothExist}{false}
}

\ifthenelse{\boolean{showPageHead}}{ %then
  \clearpairofpagestyles % No header for references pages
  }{} % No head has been set to clear

\ifthenelse{\boolean{BothExist}}{
  % then use both
  \typeout{bibliography{\econtexRoot/Add-Refs,\econtexRoot/\texname,system}}
  \bibliography{\econtexRoot/Add-Refs,\econtexRoot/\texname,system}
  % else both do not exist
}{ % maybe neither does?
  \ifthenelse{\boolean{NeitherExists}}{
    \typeout{bibliography{\texname}}
    \bibliography{\texname}}{
    % no -- at least one exists
    \ifthenelse{\boolean{AddRefsExists}}{
      \typeout{bibliography{\econtexRoot/Add-Refs,\econtexRoot/\texname}}
      \bibliography{\econtexRoot/Add-Refs,\econtexRoot/\texname}}{
      \typeout{bibliography{\econtexRoot/\texname,system}}
      \bibliography{        \econtexRoot/\texname,system}}
  } % end of picking the one that exists
} % end of testing whether neither exists
}

\ifthenelse{\boolean{Web}}{}{
  \onlyinsubfile{\captionsetup[figure]{list=no}}
  \onlyinsubfile{\captionsetup[table]{list=no}}
}

\end{document} \endinput
