\input{./econtexRoot.texinput}
\documentclass[\econtexRoot/IneqMeas]{subfiles}
\onlyinsubfile{\externaldocument{\econtexRoot/IneqMeas}} % Get xrefs -- esp to apndx -- from main file; only works if main file has already been compiled

\begin{document}

\onlyinsubfile{\setcounter{section}{1}}
\section{Baseline model: choice under risk}
\notinsubfile{\label{sec:BaselineModel}}

\par Here, I provide an explicit reformulation of the previous work done on proposing measures of income inequality using the choice under objective uncertainty literature.

\subsection{Analytical framework}

\par Denote the set of outcomes $Y = [0,\bar{y}]$. Then, the choice set is given by:

$$ L = \bigg\{ p: 2^{[0,\bar{y}]} \to \mathbb{R} \hspace{2mm} | \hspace{2mm} p(\cdot) \hspace{2mm} \text{is an income frequency distribution}  \bigg\}. $$

\par We may define a binary relation over both sets $Y,L:$

$$ \succsim_{y} \subseteq [0,\bar{y}] \times [0,\bar{y}] \subseteq \mathbb{R} \times \mathbb{R} $$

$$ \succsim L \times L. $$

\par Notice that $\succsim_y$ may be represented by a real-valued utility function. This is the social welfare $U(y)$ which is a key object of analysis in this paper. 

\subsection{Axiomatization}

\begin{ax.V}[Weak Order]
$\succsim$ is complete and transitive.
\end{ax.V}

\begin{ax.V}[Continuity]
For every $p(\cdot)$, $p*(\cdot)$, $p'(\cdot)$ $\in L$, if $p(\cdot) \succ p*(\cdot) \succ p'(\cdot) $, there exists $\alpha, \beta \in (0,1)$ such that 
$$ \alpha p + (1-\alpha)p' \succ p* \succ \beta p + (1-\beta) p' .$$ 
\end{ax.V}

\begin{ax.V}[Independence]
For every $p,p*,p' \in L$ and every $\alpha \in (0,1)$ such that 
$$ p \succsim p* \implies \alpha p + (1-\alpha)p' \succsim \alpha p^* + (1-\alpha)p'. $$
\end{ax.V}

\subsection{Expected utility representation of the ranking over income distributions}

\par The ranking over income distributions will be represented using the vNM-expected utility representation. That is,
$$ p(y) \sim \int_{0}^{\bar{y}} U(y) p(y) dy \equiv W. $$

\par Formally, note the slight modification of the vNM-EU theorem in this setting of ranking income distributions:

\begin{tm}
$\succsim \subseteq L \times L$ satisfies $(V1)$ \textbf{weak order}, $(V2)$ \textbf{continuity}, $(V3)$ \textbf{independence} if and only if there exists $U: Y \to \mathbb{R}$ such that, for every $p(y), p*(y) \in L,$
$$ p(y) \succsim p*(y) \iff \int_{0}^{\bar{y}} U(y) p(y) dy \geq \int_{0}^{\bar{y}} U(y) p*(y) dy. $$
\end{tm}




\onlyinsubfile{% Allows two (optional) supplements to hard-wired \texname.bib bibfile:
% system.bib is a default bibfile that supplies anything missing elsewhere
% Add-Refs.bib is an override bibfile that supplants anything in \texfile.bib or system.bib
\provideboolean{AddRefsExists}
\provideboolean{systemExists}
\provideboolean{BothExist}
\provideboolean{NeitherExists}
\setboolean{BothExist}{true}
\setboolean{NeitherExists}{true}

\IfFileExists{\econtexRoot/Add-Refs.bib}{
  % then
  \typeout{References in Add-Refs.bib will take precedence over those elsewhere}
  \setboolean{AddRefsExists}{true}
  \setboolean{NeitherExists}{false} % Default is true
}{
  % else
  \setboolean{AddRefsExists}{false} % No added refs exist so defaults will be used
  \setboolean{BothExist}{false}     % Default is that Add-Refs and system.bib both exist
}

% Deal with case where system.bib is found by kpsewhich
\IfFileExists{/usr/local/texlive/texmf-local/bibtex/bib/system.bib}{
  % then
  \typeout{References in system.bib will be used for items not found elsewhere}
  \setboolean{systemExists}{true}
  \setboolean{NeitherExists}{false}
}{
  % else
  \typeout{Found no system database file}
  \setboolean{systemExists}{false}
  \setboolean{BothExist}{false}
}

\ifthenelse{\boolean{showPageHead}}{ %then
  \clearpairofpagestyles % No header for references pages
  }{} % No head has been set to clear

\ifthenelse{\boolean{BothExist}}{
  % then use both
  \typeout{bibliography{\econtexRoot/Add-Refs,\econtexRoot/\texname,system}}
  \bibliography{\econtexRoot/Add-Refs,\econtexRoot/\texname,system}
  % else both do not exist
}{ % maybe neither does?
  \ifthenelse{\boolean{NeitherExists}}{
    \typeout{bibliography{\texname}}
    \bibliography{\texname}}{
    % no -- at least one exists
    \ifthenelse{\boolean{AddRefsExists}}{
      \typeout{bibliography{\econtexRoot/Add-Refs,\econtexRoot/\texname}}
      \bibliography{\econtexRoot/Add-Refs,\econtexRoot/\texname}}{
      \typeout{bibliography{\econtexRoot/\texname,system}}
      \bibliography{        \econtexRoot/\texname,system}}
  } % end of picking the one that exists
} % end of testing whether neither exists
}

\ifthenelse{\boolean{Web}}{}{
  \onlyinsubfile{\captionsetup[figure]{list=no}}
  \onlyinsubfile{\captionsetup[table]{list=no}}
}
\end{document}	\endinput

