\input{./econtexRoot.texinput}
\documentclass[\econtexRoot/IneqMeas]{subfiles}
\onlyinsubfile{\externaldocument{\econtexRoot/IneqMeas}} % Get xrefs -- esp to apndx -- from main file; only works if main file has already been compiled

\begin{document}

\onlyinsubfile{\setcounter{section}{3}}
\section{Extending the model: choice under ambiguity}
\notinsubfile{\label{sec:ExtendedModel}}

\par Consider the excerpt from \cite{g09}, which may serve as a precursor to the modeling choices that will be made to characterize a ranking over wealth distributions:

\begin{quote}


To make sure that we understand the structure, observe that there are two sources of uncertainty: the choice of the state $s$, which is sometimes referred to as \say{subjective uncertainty}, because no objective probabilities are given on it, and the choice of $x$, which is done with objective probabilities once you chose your act and Nature chose a state. Specifically, if you choose $f \in F$ and Nature chooses $s \in S$, a roulette wheel is spun, with distribution $f(s)$ over the outcomes $X$, so that your probability to get outcome $x$ is $f(s)(x)$.

\end{quote}

\subsection{Analytical framework}

\par Denote the set of outcomes $Y = [0,\bar{y}]$. Then, the choice set is given by:

$$ L = \bigg\{ p: 2^{[0,\bar{y}]} \to \mathbb{R} \hspace{2mm} | \hspace{2mm} p(\cdot) \hspace{2mm} \text{is an income frequency distribution}  \bigg\}. $$

\par We may define a binary relation over both sets $Y,L:$

$$ \succsim_{y} \subseteq [0,\bar{y}] \times [0,\bar{y}] \subseteq \mathbb{R} \times \mathbb{R} $$

$$ \succsim L \times L. $$

\par Notice that $\succsim_y$ may be represented by a real-valued utility function. This is the social welfare $U(y)$ which is a key object of analysis in this paper. 

\subsubsection*{Preliminary remarks}

We are concerned with the comparison of two frequency distributions $f(w)$ of an outcome $w$ which we refer to as wealth. We seek to use the notion of \textit{uncertainty aversion} as an analogy to the use of the notion of risk aversion in the characterization of a ranking over income distributions.

The presence of both objective and subjective uncertainty is at the heart of this analysis. Section 2 covered the analysis for objective uncertainty. Thus, the wealth frequency distribution $f(w)$ must be formalized in this abstract setting so that it explicitly captures both forms of uncertainty. Namely, each wealth distribution is associated with some relevant, underlying state space $S$ (the source of subjective uncertainty), and its objective component $p(y) \in L$. WIth this in mind, we work under the following assumption on the functional form of wealth distributions for the remainder of this paper:

\begin{assu}
Each wealth distribution $f(w)$ can be written as $p(s)y$.
\end{assu}


\subsection{Axiomatization}

\begin{ax.AA}[Weak Order]
$\succsim$ is complete and transitive.
\end{ax.AA}

\begin{ax.AA}[Continuity]
For every $f,g,h \in F$,  if $f \succ g \succ h $, there exists $\alpha, \beta \in (0,1)$ such that 
$$ \alpha f + (1-\alpha) h \succ g \succ \beta f + (1-\beta) h .$$ 
\end{ax.AA}

\begin{ax.C}[C-Independence]
For every $f,g \in F$, every constant $h \in F$ and every $\alpha \in (0,1)$,
$$ f \succsim g \iff \alpha f + (1-\alpha)h \succsim \alpha g + (1-\alpha)h. $$
\end{ax.C}

\begin{ax.AA}[Monotonicity]
For every $f,g \in F$, $f(s) \geq g(s)$ for all $s \in S$ implies $f \geq g$.
\end{ax.AA}

\begin{ax.AA}[Non-trivality]
There exists $f,g \in X$ such that $f \succ g$.
\end{ax.AA}

\begin{ax.U}[Uncertainty Aversion]
For every $f,g \in F$, if $f \sim g$, then, for every $\alpha \in (0,1)$,
$$ \alpha f + (1-\alpha)g \succsim f. $$
\end{ax.U}

\subsection{Expected utility representation of the ranking over wealth distributions}

\par  Finally, we have the representation theorem by \cite{gs89}.



\begin{tm}
$\succsim $ satisfies AA1, AA2, C-Independence, AA4, AA5, and Uncertainty aversion if and only if there exists a closed and convex set of probabilities on $S$, $C \subset \Delta(S)$, and a non-constant function $U: Y \to \mathbb{R}$ such that, for every $f, f^* \in F,$
$$ f \succsim f^* \iff \min_{\lambda \in \Delta(S)} \int_{S} (\mathbb{E}_{p(s)} u) d \lambda  \geq \min_{\lambda \in \Delta(S)} \int_{S} (\mathbb{E}_{p^{*}(s)} u) d \lambda.$$
\end{tm}

From here on out, we assume that wealth distributions will be ranked according to:

$$ W' \equiv \min_{\lambda \in \Delta(S)} \int_{S} (\mathbb{E}_{p(s)} u) d \lambda $$

$$ = \min_{\lambda \in C} \int_{S} \int_{0}^{\bar{y}} p(s)(y) U(y) dy d\lambda. $$


\onlyinsubfile{% Allows two (optional) supplements to hard-wired \texname.bib bibfile:
% system.bib is a default bibfile that supplies anything missing elsewhere
% Add-Refs.bib is an override bibfile that supplants anything in \texfile.bib or system.bib
\provideboolean{AddRefsExists}
\provideboolean{systemExists}
\provideboolean{BothExist}
\provideboolean{NeitherExists}
\setboolean{BothExist}{true}
\setboolean{NeitherExists}{true}

\IfFileExists{\econtexRoot/Add-Refs.bib}{
  % then
  \typeout{References in Add-Refs.bib will take precedence over those elsewhere}
  \setboolean{AddRefsExists}{true}
  \setboolean{NeitherExists}{false} % Default is true
}{
  % else
  \setboolean{AddRefsExists}{false} % No added refs exist so defaults will be used
  \setboolean{BothExist}{false}     % Default is that Add-Refs and system.bib both exist
}

% Deal with case where system.bib is found by kpsewhich
\IfFileExists{/usr/local/texlive/texmf-local/bibtex/bib/system.bib}{
  % then
  \typeout{References in system.bib will be used for items not found elsewhere}
  \setboolean{systemExists}{true}
  \setboolean{NeitherExists}{false}
}{
  % else
  \typeout{Found no system database file}
  \setboolean{systemExists}{false}
  \setboolean{BothExist}{false}
}

\ifthenelse{\boolean{showPageHead}}{ %then
  \clearpairofpagestyles % No header for references pages
  }{} % No head has been set to clear

\ifthenelse{\boolean{BothExist}}{
  % then use both
  \typeout{bibliography{\econtexRoot/Add-Refs,\econtexRoot/\texname,system}}
  \bibliography{\econtexRoot/Add-Refs,\econtexRoot/\texname,system}
  % else both do not exist
}{ % maybe neither does?
  \ifthenelse{\boolean{NeitherExists}}{
    \typeout{bibliography{\texname}}
    \bibliography{\texname}}{
    % no -- at least one exists
    \ifthenelse{\boolean{AddRefsExists}}{
      \typeout{bibliography{\econtexRoot/Add-Refs,\econtexRoot/\texname}}
      \bibliography{\econtexRoot/Add-Refs,\econtexRoot/\texname}}{
      \typeout{bibliography{\econtexRoot/\texname,system}}
      \bibliography{        \econtexRoot/\texname,system}}
  } % end of picking the one that exists
} % end of testing whether neither exists
}

\ifthenelse{\boolean{Web}}{}{
  \onlyinsubfile{\captionsetup[figure]{list=no}}
  \onlyinsubfile{\captionsetup[table]{list=no}}
}

\end{document} \endinput
