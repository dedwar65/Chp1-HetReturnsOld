\input{./econtexRoot.texinput}
\documentclass[\econtexRoot/IneqMeas]{subfiles}
\onlyinsubfile{\externaldocument{\econtexRoot/IneqMeas}} % Get xrefs -- esp to apndx -- from main file; only works if main file has already been compiled

\begin{document}

\onlyinsubfile{\setcounter{section}{2}}
\section{Motivation}
\notinsubfile{\label{sec:Motivation}}


\par For the purposes of this paper, we will assume that ranking wealth distributions is a non-trivially distinct exercise from the previous case of ranking income distributions. The motivation for this can be approached from many directions. I will make note of them here, but a more rigorous defense of these points is beyond the scope of this paper.

\subsection{A statistical argument}

\par It is well-known that income data is more reliably collected than wealth data. Consider a quote from \cite{sz16} which provides an estimate for wealth inequality using available administrative data:

\begin{quote}

Because of the lack of administrative data on wealth, none of the existing sources offer a definitive estimate. We see our paper as an attempt at using the most comprehensive administrative data currently available, but one that ought to be improved in at least two ways: (i) by using additional information already available at the Statistics of Income division of the IRS, and (ii) new data that the US Treasury could collect at low cost. A modest data collection effort would make it possible to obtain a better picture of the joint distributions of wealth, income, and saving, a necessary piece of information to evaluate proposals for consumption or wealth taxation.

\end{quote}

\par So it is clear: there are differences in the collection of income data and wealth data. But is it enough to justify introducing subjective uncertainty in the exercise of comparing wealth distributions?

\par Consider next a quote from \cite{bks16} which provides a justification for their use of administrative data and their measurement of income and wealth inequality:

\begin{quote}

In general, administrative data should provide better estimates of top income and wealth shares, because traditional random household surveys suffer from under- representation of wealthy families. Unlike most other household surveys, the SCF is designed to overcome the underrepresentation problem, because administrative data are used to select the sample, and rigorous targeting and accounting for wealthy family participation assures those families are properly represented in the survey data.

\end{quote}

\par Not only is \say{good} data on income more accessible, but the estimation of wealth shares typically employs some combination of administrative and survey data. But survey data is by, quite literally, a subjective measure: it implicitly relies on individual's truthful and accurate reporting of their own levels of wealth.

\subsection{A thought experiment}

\par Country A has $60$ individuals, each of either one of two races, black or white. Suppose that
there are $10$ black and $50$ white individuals. There is an initial wealth distribution $f(w)$. One
can consider the total level of wealth, $\bar{w} = \sum_{i=1}^{60} w_i f(w_i)$, as well as
the total level of wealth within each racial group, $\bar{w}_b$ and $\bar{w}_w$.

\par A policymaker if tasked with comparing this initial distribution of wealth $f(w)$ with the
distribution $f'(w)$ reached by the following redistribution of wealth:

\begin{quote}

\say{Increase wealth of white individuals by 2 units, and decrease wealth of black individuals
by 9 units}

\end{quote}

\par The redistributive policy outlined above suggests redistributing wealth from black individuals who are wealthier within group to poorer white individuals within group. If the policymaker takes a \textit{Utilitarian} view of social welfare, then they should favor this policy since:

$$ 9 * 10 < 2 * 50 \implies f(\bar{w}_b - 10)U(\bar{w}_b -10) + f(\bar{w}_w +2)U(\bar{w}_w + 2) > f(\bar{w}_b )U(\bar{w}_b) + f(\bar{w}_w)U(\bar{w}_w)   $$

\par However, it is not unreasonable to suspect that a policy such as this would not garner much support from individuals in practice, considering racial differences in wealth holdings in the U.S. This suggests that the choice under objective uncertainty model may be lacking in its ability to implement favorable wealth distributive policies.

\par A notion of subjective uncertainty can be incorporated into this thought experiment in a straightforward way: denote the finite state space S =\{black, white\}. From here, using a model which allows for one to define a notion of uncertainty aversion is a reasonable deviation from the aforementioned case.

\subsection{A non-economic argument}

\par Though \say{non-economic} arguments seem to be the least compelling up front, the following statements of the late philosopher John Rawls may convince readers that this is the most compelling argument for the modeling choices to come.

\par \cite{jr71} \textit{theory of distributive justice} is interested in the problem of getting a group of people with different circumstances and motives to agree on a \say{social contract}; that is, an agreement on a system of governing for which all members of the group must abide by.

\par He proposes that this problem should be approached as if those deciding on the governance of society are behind a \textit{veil of ignorance}: decision-makers are ignorant of their own circumstances. He then proposes two principles that may accompany this veil of ignorance in the delivery of justice for institutions which govern our society. Consider Rawls' final statement of the two principles of justice for institutions:

\begin{enumerate}
\item Each person is to have an equal right to the most extensive total system of equal basic liberties compatible with a similar system of liberty for all.

\item Social and economic inequalities are to be arranged so that they are both:

\begin{enumerate}
\item to the greatest benefit of the least advantaged, consistent with the just savings principle, and
\item attached to offices and positions open to all under conditions of fair equality of opportunity.
\end{enumerate}
\end{enumerate}

\par But how is this related to the social planner's decision problem of ranking wealth distributions? The link is that there is some underlying notion of a social contract implicit in any concept of wealth. That is, the outcome wealth is incoherent without some underlying notion of ownership! Individuals must agree on what it means to own something, and must respect that some individuals own more and/or less than they do.

\par If the social planner takes this fact about wealth seriously, then this may suggest alternative modeling choices than those made in the income context. Consider another quote from Rawls, which is, quite literally, the intuitive counterpart to the modeling choices I make later in the characterization of a comparison over wealth distributions:

\begin{quote}

The maximin rule tells us to rank alternatives by their worst possible outcomes: we are to adopt the alternative the worst outcome of which is superior to the worst outcomes of the others\footnote{Coincidentally, Rawls' second principle of justice for institutions is often considered by economists to be interchangeable with the maximin principle, despite his belief that it is \say{undesirable to use the same name for two things that are so distinct.}.}.

\end{quote}



\onlyinsubfile{% Allows two (optional) supplements to hard-wired \texname.bib bibfile:
% system.bib is a default bibfile that supplies anything missing elsewhere
% Add-Refs.bib is an override bibfile that supplants anything in \texfile.bib or system.bib
\provideboolean{AddRefsExists}
\provideboolean{systemExists}
\provideboolean{BothExist}
\provideboolean{NeitherExists}
\setboolean{BothExist}{true}
\setboolean{NeitherExists}{true}

\IfFileExists{\econtexRoot/Add-Refs.bib}{
  % then
  \typeout{References in Add-Refs.bib will take precedence over those elsewhere}
  \setboolean{AddRefsExists}{true}
  \setboolean{NeitherExists}{false} % Default is true
}{
  % else
  \setboolean{AddRefsExists}{false} % No added refs exist so defaults will be used
  \setboolean{BothExist}{false}     % Default is that Add-Refs and system.bib both exist
}

% Deal with case where system.bib is found by kpsewhich
\IfFileExists{/usr/local/texlive/texmf-local/bibtex/bib/system.bib}{
  % then
  \typeout{References in system.bib will be used for items not found elsewhere}
  \setboolean{systemExists}{true}
  \setboolean{NeitherExists}{false}
}{
  % else
  \typeout{Found no system database file}
  \setboolean{systemExists}{false}
  \setboolean{BothExist}{false}
}

\ifthenelse{\boolean{showPageHead}}{ %then
  \clearpairofpagestyles % No header for references pages
  }{} % No head has been set to clear

\ifthenelse{\boolean{BothExist}}{
  % then use both
  \typeout{bibliography{\econtexRoot/Add-Refs,\econtexRoot/\texname,system}}
  \bibliography{\econtexRoot/Add-Refs,\econtexRoot/\texname,system}
  % else both do not exist
}{ % maybe neither does?
  \ifthenelse{\boolean{NeitherExists}}{
    \typeout{bibliography{\texname}}
    \bibliography{\texname}}{
    % no -- at least one exists
    \ifthenelse{\boolean{AddRefsExists}}{
      \typeout{bibliography{\econtexRoot/Add-Refs,\econtexRoot/\texname}}
      \bibliography{\econtexRoot/Add-Refs,\econtexRoot/\texname}}{
      \typeout{bibliography{\econtexRoot/\texname,system}}
      \bibliography{        \econtexRoot/\texname,system}}
  } % end of picking the one that exists
} % end of testing whether neither exists
}

\ifthenelse{\boolean{Web}}{}{
  \onlyinsubfile{\captionsetup[figure]{list=no}}
  \onlyinsubfile{\captionsetup[table]{list=no}}
}

\end{document} \endinput
